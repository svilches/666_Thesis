Photonics West:

Optical biopsy is collective name for the advanced endomicroscopy methods allowing real-time, in-situ diagnosis of tissue malignancies, without the need of excision and histopathological analysis. Numerous techniques, such as endocytoscopy and fluorescence imaging, have already found their way into the clinical armamentarium over the last years; others, such as confocal laser endomicroscopy,1 Raman spectroscopy2 or optical coherence tomography (OCT),3, 4 are expected to enter routine clinical use in the imminent future. At the current state of the art, none of these imaging modalities can yet match the selectivity and specificity of traditional biopsy. This is why multi-modal imaging, which can provide complete tissue characterization through complementary modalities, will be the key to the success of optical biopsy.




Simon thesis:

Endoscopic methods play an increasingly important role in modern medicine. In general, the goal of endoscopy is to obtain as much information as possible for a viable diagnosis. To further improve diagnostics, there is a clear tendency towards multi-modal imaging by combining different imaging techniques. Making use of microscopy and optical coherence tomography in a single device, is a promising approach to gather additional information and to thereby improve clinical diagnostics. 

This thesis presents a MEMS-based bimodal endoscopic probe which makes use of the aforementioned imaging modalities. The probe features a two-level design by which it is possible to easily integrate the two modalities. Furthermore, the two-level approach allows the adjustment of the two modalities to their respective requirements.

Optical characterization of two prototypes shows that the resolutions achieved with this silicon endomicroscope are very close to the design values, confirming the applicability of MEMS fabrication techniques for the construction of endoscopic probes.






Der demographische Wandel der Gesellschaften mit einer immer älter werdenden Weltbevölkerung stellt unsere medizinische Versorgung vor immer neue Herausforderungen \cite{fendrich2007}. Neben der zunehmenden Wichtigkeit von spezialisierten Therapien und Medikamenten steigt auch das Bedürfnis nach besseren und weniger belastenden diagnostischen Hilfsmitteln bei der Früherkennung von Krankheiten. Hierbei wird das Ziel verfolgt, invasive Diagnosemethoden durch nicht oder wenig invasive Verfahren zu ersetzen, um so die Belastung der Patienten zu reduzieren.

Ein Beispiel für invasive Diagnosemethoden ist die Biopsie. Diese wird beispielsweise in der Krebsdiagnostik angewendet. Bei einer Biospie wird eine Gewebeprobe entnommen und diese anschließend in einem Labor auf Veränderungen untersucht \cite{huang1996}. Die Entnahme der Proben ist für die Patienten meist sehr unangenehm und je nach Ort im Körper auch sehr belastend. Inzwischen konnten die Verfahren für einige Körperregionen deutlich verbessert werden. So können Biopsien in Hohlorganen, wie zum Beispiel dem Gastrointestinaltrakt, durch die Verwendung von Endoskopen deutlich schonender durchgeführt werden \cite{vilmann2006}. Für finale Aussagen über mögliche Gewebeveränderungen müssen allerdings weiterhin Gewebeproben an den entsprechenden Stellen entnommen werden \cite{vilmann2006}. Um den Grad der Belastung bei der Verwendung von Endoskopen noch weiter zu verringern, werden Verfahren entwickelt, die detaillierte Aussagen über die Gewebebeschaffenheit ohne die Entnahme einer Probe ermöglichen.

Ein Ansatz, um endoskopbasierte Biopsien zu verdrängen ist, mit Hilfe optischer Verfahren zusätzliche Informationen über die Gewebebeschaffenheit zu sammeln und diese Informationen im Diagnoseprozess heranzuziehen. Das Ziel ist die herkömmliche endoskopbasierte Biopsie durch eine optische Biopsie zu ersetzen \cite{shukla2011}. Besonders interessant sind dabei Verfahren, die mit dem endoskopischen Ansatz kombinierbar sind. Beispiele für optische Verfahren, die bereits endoskopisch Anwendung finden, sind die optische Kohärenztomographie \cite{huang1991,sergeev1997,bouma2000,sivak2000,fujimoto2003,tran2004,liu2011}, die konfokale Laser Scanning Fluoreszenz Mikroskopie \cite{white1987,kiesslich2004,polglase2005} und die Raman-Spektroskopie \cite{bergholt2011}. Durch diese Verfahren ist es unter bestimmten Umständen möglich, auf die Gewebeentnahme zur Diagnose zu verzichten.

Auch an der \textit{Gisela und Erwin Sick Professur für Mikrooptik} des \textit{Instituts für Mikrosystemtechnik} (kurz: IMTEK) werden endoskopische Sonden für die erweiterte Gewebeuntersuchung erforscht. Besonderes Augenmerk wird dabei auf die Implementierung kompakter, mikrotechnisch hergestellter Sonden für die optische Kohärenztomographie (engl.: \textit{optical coherence tomography}, kurz: OCT) gelegt. OCT ist eine interferometrische Methode, die es mittels infrarotem Licht ermöglicht, Tiefenschnitte von streuenden Proben, wie zum Beispiel Geweben, zu erzeugen \cite{huang1991,fujimoto2003}. Die erhaltenen Bilder sind mit Ultraschallaufnahmen vergleichbar. Mit OCT kann allerdings eine höhere Auflösung als mit Ultraschall erreicht werden \cite{drexler2008}. Gleichzeitig ist die Eindringtiefe des infraroten Lichts in die streuenden Proben deutlich geringer. Die mit OCT erzeugten Tiefenbilder können genutzt werden, um Veränderungen in oberflächennahen Geweben, wie zum Beispiel Schleimhäuten, zu erkennen \cite{sergeev1997}. 

Im Folgenden wird nun der Stand der Technik für auf OCT basierende, endoskopische Sonden vorgestellt. Anschließend wird das Ziel des Forschungsprojekts HYAZINT und die Einbettung dieser Arbeit in dieses Projekt erläutert. Am Ende des Kapitels wird die exakte Zielsetzung der Arbeit zusammengefasst.



The presented work builds on the concept of a two layer, MEMS based silicon optical bench [1] that combines white light microscopy with OCT on a single integrated silicon microbench. In contrast to other approaches [2], the combination of white light microscopy with OCT is realized without the need of a coherent fiber bundle. With this design, the inherent drawbacks of such fiber bundles can be avoided, which are, for example, low light troughput, multi-modal coupling and poor resolution for a given field of view as stated in [3]. Furthermore, with the two level approach we can implement modalities with different requirements regarding the numerical aperture of the optical system as it is the case for white light microscopy and OCT.



The aim of this work is to design and test a miniaturized OCT microscope as a component of a multi-modal endoscopic probe. This probe consists of two spectrally-separated optical paths that run partially in parallel through a micro-optical bench, as depicted in \autoref{fig:bimodalSketch} \cite{Kretschmer}. This approach allows independent design of the optical parameters of the two imaging modalities -- such as the numerical aperture (NA) or depth of field -- while still providing a geometrical overlap of the two acquired images. An integrated tubular piezoelectric fiber scanner is used to perform en face scanning required for three dimensional OCT measurements. This scanning engine has an outer diameter of \SI{0.9}{\milli\meter} and a length of \SI{9}{\milli\meter}, and features custom fabricated \SI{10}{\micro\meter} thick polyimide flexible interconnect lines to address the four piezoelectric electrodes.


\item The scanner, electrical connections and optics should fit in a channel with a $\SI{1}{\milli\meter} \times \SI{1}{\milli\meter}$ cross section located in the lower level of a multimodal bench. This way the total cross section of the endoscope can be kept below $3 \times \SI{2}{\milli\meter^2}$). 
\item Its length should be minimized to allow its integration in flexible-head endoscopes.
\item The field of view should be maximized for a 2 mm diameter objective lens, that is shared with the endomicroscopy beam path.
\item The scanning speed should be adequate for the sampling rates characteristic of SD-OCT ($\sim \SI{100}{\kilo\hertz} $).
\end{itemize}


\paragraph{Optical Requirements}

\begin{itemize}
\item The microscopy and OCT imaging fields should be coaxial to avoid parallax errors. 
\item The OCT field should be telecentric to avoid field curvature distortions.
\item The lateral resolution and depth of field should be adequate for OCT i.e. with numerical aperture ranging from 0.02 to 0.05.
\item The backreflections inside the probe should be minimized to reduce any loss of contrast and penetration depth.
\end{itemize}

The main challenge of this work is to design an OCT scanning mechanism compact enough to be placed in a thin, buried channel of a multimodal probe.  Although it is theoretically possible to keep a scanner at the proximal end of the endoscope and use a coherent fiber bundle as a relay, there are inherent drawbacks of this method when applied for OCT, such as low light throughput, cross-talk and mechanical rigidity \cite{Ford2009}. 

Another challenging requirement is the superposition of the images acquired by the different modalities. If the optical axes are not coaxial, the fields will be shifted and tilted due to parallax error --- which gains importance at the small working distances common in endoscopy.

To overcome these problems, and taking into account the above-mentioned requirements, we propose a design based on the HYAZINT multimodal probe \cite{Blattmann2016}. By creating a two layer microbench, it is possible to bury the OCT resonant fiber scanner in the bottom level and merge both modalities in the top level using a dichroic beamsplitter. An schematic of this mechanism can be seen in \autoref{fig:bimodalSketch}.

The base of the microbench with dimensions of $13 \times 2 \times \SI{1}{\milli\meter^3} $ is realized by standard silicon bulk micromachining. On the top layer, the bench accommodates the full field imaging optics that consists of a dichroic beamsplitter cube with dimensions of $2\times 2 \times  \SI{2}{\milli\meter^3}$ to separate the two beam paths and two plano-convex lenses with \SI{2}{\milli\meter} diameter, which form a full field microscope. To achieve a highly compact opto-mechanical design, the components of the OCT beam path are buried within a cavity in the base of the micro bench. On the bottom layer a gradient index lens (GRIN lens) with a diameter of \SI{350}{\micro\meter} is directly glued to the tip of a \SI{80}{\micro\meter} single mode fiber to collimate the infrared light of the OCT system with a center wavelength of $\lambda_o = \SI{1311}{\nano\meter}$. A spiral scanning of the OCT beam path is achieved by an angular scanner implemented using a piezoelectric tube actuator.

This actuator, called resonant fiber scanner, is able to scan a collimated beam by more than $\pm \SI{5}{\degree}$ by mechanically amplifying the subtle vibration of a piezoelectric actuator.  An objective lens focuses the beam on the tissue and transforms the angular displacement into a translation. By driving the scanner in two axes with two sinusoids at different phases, it is possible to sample a 2D area of the object in a spiral fashion \cite{Seibel2006}, as explained in detail in Section \ref{sec:Optical}.

The rest of this chapter shows the design and development of the OCT imaging path for the multi-modal probe. However, in order to independently test the behavior of the OCT scanner and optics, a single modality probe was fabricated as a demonstrator. Both systems are mechanically and optically equivalent -- the only difference is the presence of the beam splitter. 
For completeness, both multi-mode and single-mode optical systems are described.