 % -*- encoding: UTF8 -*-
%
%%*****************************************************************************
%%									                         			Einführung									 
%%*****************************************************************************

\chapter{Introduction}
\label{Ch:Introduction}
%%*****************************************************************************


\section{Motivation}

Optical biopsy

Simultaneous

Multimode as key technology (include other modalities)

The next challenge lies in the size: The external diameter of an endoscope constrains its field of application. For example, in cystoscopy (endoscopy of the urinary bladder), probes with small diameter (under 5 mm) reduce the pain and trauma to urethra.





\clearpage
\section{State of the Art}
\begin{figure}[h!]
      \centering
      \includegraphics[width=10cm,draft]{figures/foo.png}
      \caption{Simon bench, Tobias Scanner, Seibel scanner}
      %\label{}
\end{figure}


\clearpage
\section{Approach of this thesis}
The presented work builds on the concept of a two layer, MEMS based silicon optical bench [1] that combines white light microscopy with OCT on a single integrated silicon microbench. In contrast to other approaches [2], the combination of white light microscopy with OCT is realized without the need of a coherent fiber bundle. With this design, the inherent drawbacks of such fiber bundles can be avoided, which are, for example, low
light troughput, multi-modal coupling and poor resolution for a given field of view as stated in [3]. Furthermore, with the two level approach we can implement modalities with different requirements regarding the numerical aperture of the optical system as it is the case for white light microscopy and OCT.