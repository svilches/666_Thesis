% encoding:  UTF8

%%%%%%%%%%%%%%%%%%%%%%%%%%%%%%%%%%%%%%%%%%%%%%%%%%%%%%%%%%
%%                                                                                                             		                                      %%
%%                                                Default settings																				%%
%%                                                                                                                                                       %%
%%%%%%%%%%%%%%%%%%%%%%%%%%%%%%%%%%%%%%%%%%%%%%%%%%%%%%%%%%

\NeedsTeXFormat{LaTeX2e}[2005/12/01]

%% Moegliche Optionen: diejenigen der Klasse scrbook ausser titlepage

%% deutsche DA:
%\documentclass[%a4paper, 
%                          master,                                     %% Typ der Arbeit: diplom, bachelor oder master
%                          12pt,                                                      %% Schriftgroesse
%                          twoside,                                                  %% zweiseitiges Layout
%                          BCOR10mm,                                           %% Bindekorrektur 10 mm
%                          %liststotoc,nomtotoc,                   %% Aufnahme der div. Verzeichnisse ins Inhaltsverzeichnis
%													bibtotoc, 
%                          %pointlessnumbers,                                   %% Ueberschriftnummer. ohne angehaengtem Punkt
%                          english,ngerman,                                      %% Alternativspr. Englisch, Dokumentspr. Deutsch
%                          final,                                                        %% Endversion; draft fuer schnelles Kompilieren
%                          ]{IMTEKda}

% ***  Englisch mit dt. Vorspann:   
    \documentclass[master,12pt,twoside,BCOR10mm,pointlessnumbers,ngerman,english,englishpreamble]{IMTEKda}

% *** Labels anzeigen zur Korrektur
    %\usepackage{showkeys} %% Labels verschwinden mit der Klassenoption final

\usepackage{babel}                  %% Sprachen-Unterstuetzung
\usepackage{calc}                    %% ermoeglicht Rechnen mit Laengen und Zaehlern


\usepackage[T1]{fontenc} 
\usepackage[utf8]{inputenc}            %% UTF8-Codierung !!!
  
% *** For citation in Chicago Style

\usepackage{chicago} %\usepackage{chicagoa}
%\nocite{*} 	% If this command is used the whole bibliography will be printed, 
				% if not, only the cited references
\renewcommand{\cite}{\shortcite}
\bibliographystyle{chicago}



\usepackage{amsmath,amssymb}   %% zusaetzliche Mathe-Symbole

\usepackage{lmodern}                     %% type1-taugliche CM-Schrift als Variante zur "normalen" EC-Schrift

%% Paket fuer bibtex-Datenbanken
%\usepackage[comma,numbers,sort&compress]{natbib}

\usepackage{babelbib}         %% korrekte Sprache in Bibliographieeintraegen
%\bibliographystyle{plainnat}   %% Formatierung Bibliographie ohne babelbib
%\bibliographystyle{babplain}  %% Formatierung Bibliographie mit babelbib

%\bibliographystyle{babunsrt}     %% erzeugt eine Bibliographie mit unsortierten Einträgen, d.h. in Reihenfolge der Zitate 
                                               %% - für korrekte Reihenfolge sind mehrere Latex-Durchläufe nötig ! 
												%% Außerdem ermöglicht babunsrt im Vergleich zu unsrtnat die Formatierung von Einträgen in weiteren Sprachen als nur Englisch.

\newcommand{\tabheadfont}[1]{\textbf{#1}}        %% Tabellenkopf in Fett
\usepackage{booktabs}                                            %% Befehle fuer besseres Tabellenlayout
\usepackage{longtable}                                            %% umbrechbare Tabellen
%\usepackage{array}                                               %% zusaetzliche Spaltenoptionen

%% umfangreiche Pakete fuer Symbole wie \micro, \ohm, \degree, \celsius etc.
\usepackage{textcomp,gensymb}

%\usepackage{SIunits}            %% Korrektes Setzen von Einheiten
\usepackage{units}                  %% Variante fuer Einheiten

%% Hyperlinks im Dokument; muss als eines der letzten Pakete geladen werden
\usepackage[pdfstartview=FitH,                     % Oeffnen mit fit width
                   breaklinks=true,                         % Umbrueche in Links, nur bei pdflatex default
                   bookmarksopen=true,                % aufgeklappte Bookmarks
                   bookmarksnumbered=true,        % Kapitelnummerierung in bookmarks
                   pdfprintscaling=None,                % Default-Einstellung zum Drucken: nicht skaliert
                   pdfduplex=DuplexFlipLongEdge, % Default-Druck-Einstellung: Duplex
                   ]{hyperref}

%% Bei Anklicken der im pdf-Dokument erzeugten Links springt der PDF-Viewer hiermit auf das
%% Bild / die Tabelle selbst
\usepackage[all]{hypcap}
 
% *** Um keine SANSSERIF Schriften fuer Ueberschriften zu verwenden:  ***
      %\setkomafont{sectioning}{\normalfont\normalcolor\bfseries}

%  *** Fuer kleinere Bild- und Tabellenunterschriften:
      %\addtokomafont{caption}{\footnotesize}

% *** Um abgekuerzte Abbildungs- und Tabellenbezeichnung mit \autoref zu erhalten:
      %\addto{\extrasngerman}{\renewcommand*{\figureautorefname}{Abb.}}
      %\addto{\extrasngerman}{\renewcommand*{\tableautorefname}{Tab.}}