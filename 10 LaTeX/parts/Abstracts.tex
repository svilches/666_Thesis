%%%%%%%%%%%%%%%%%%%%%%%%%%%%%%%%%%%%%%%%%%%%%%%%%%%%%%%%%%%%%%%%%%%%%%%%%%%%%%%%%
%%%%%                                                                Deutsche Kurzusammenfassung	                      																				
%%%%%%%%%%%%%%%%%%%%%%%%%%%%%%%%%%%%%%%%%%%%%%%%%%%%%%%%%%%%%%%%%%%%%%%%%%%%%%%%%
%%%%%
%%%%%
\frontmatter
\maketitle
\cleardoublepage\phantomsection\pdfbookmark{\abstractname}{abstract} %% fuegt ersten Abstract in die Bookmarks ein

\begin{otherlanguage}{ngerman}   %... falls Restdokument auf English...
\begin{abstract}
  Weit hinten, hinter den Wortbergen, fern der Länder Vokalien und Konsonantien leben die Blindtexte. Abgeschieden wohnen sie in Buchstabhausen an der Küste des Semantik, eines großen Sprachozeans. Ein kleines Bächlein namens Duden fließt durch ihren Ort und versorgt sie mit den nötigen Regelialien. Es ist ein paradiesmatisches Land, in dem einem gebratene Satzteile in den Mund fliegen. Nicht einmal von der allmächtigen Interpunktion werden die Blindtexte beherrscht – ein geradezu unorthographisches Leben. Eines Tages aber beschloß eine kleine Zeile Blindtext, ihr Name war Lorem Ipsum, hinaus zu gehen in die weite Grammatik. Der große Oxmox riet ihr davon ab, da es dort wimmele von bösen Kommata, wilden Fragezeichen und hinterhältigen Semikoli, doch das Blindtextchen ließ sich nicht beirren. Es packte seine sieben Versalien, schob sich sein Initial in den Gürtel und machte sich auf den Weg. Als es die ersten Hügel des Kursivgebirges erklommen hatte, warf es einen letzten Blick zurück auf die Skyline seiner Heimatstadt Buchstabhausen, die Headline von Alphabetdorf und die Subline seiner eigenen Straße, der Zeilengasse. Wehmütig lief ihm eine rhetorische Frage über die Wange, dann setzte es seinen Weg fort. Unterwegs traf es eine Copy.
  
	\bigskip\par
  \textbf{Stichwörter:} a, b, c, d, e, f
\end{abstract}
\end{otherlanguage}

%%%%%%%%%%%%%%%%%%%%%%%%%%%%%%%%%%%%%%%%%%%%%%%%%%%%%%%%%%%%%%%%%%%%%%%%%%%%%%%%%
%%%%%                                                                 English Abstract												
%%%%%%%%%%%%%%%%%%%%%%%%%%%%%%%%%%%%%%%%%%%%%%%%%%%%%%%%%%%%%%%%%%%%%%%%%%%%%%%%%
%%%%%
%%%%%
%\begin{otherlanguage}{english}
\begin{abstract}
  Endoscopic methods play an increasingly important role in modern medicine. In general, the goal of endoscopy is to obtain as much information as possible for a viable diagnosis. To further improve diagnostics, there is a clear tendency towards multi-modal imaging by combining different imaging techniques. Making use of microscopy and optical coherence tomography in a single device is a promising approach to gather additional information and to thereby improve clinical diagnostics. 

The main goal of this work is the realization of a novel single-fiber-based scanning scheme enabling simultaneous 3D OCT in combination with full field microscopy. This integrated tubular piezoelectric fiber scanner is used to perform en face scanning required for three dimensional OCT measurements. The complete scanning engine has an outer diameter of \SI{0.9}{\milli\meter} and a length of \SI{9}{\milli\meter}, and features custom fabricated \SI{10}{\micro\meter} thick polyimide flexible interconnect lines to address the four piezoelectric electrodes.

Optical and mechanical characterization of the scanner, implemented in a single modality demonstrator probe, show a high correlation with the analytical calculations and simulations. Thus, the workflow described in this work constitutes a blue print for a wide range of future scanning imaging probes.

\bigskip\par
  \textbf{Keywords:} Endoscopic probe, bimodal, optical coherence tomography, fiber scanner.
\end{abstract}
%\end{otherlanguage}