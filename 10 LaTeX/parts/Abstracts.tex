%%%%%%%%%%%%%%%%%%%%%%%%%%%%%%%%%%%%%%%%%%%%%%%%%%%%%%%%%%%%%%%%%%%%%%%%%%%%%%%%%
%%%%%                                                                Deutsche Kurzusammenfassung	                      																				
%%%%%%%%%%%%%%%%%%%%%%%%%%%%%%%%%%%%%%%%%%%%%%%%%%%%%%%%%%%%%%%%%%%%%%%%%%%%%%%%%
%%%%%
%%%%%
\frontmatter
\maketitle
\cleardoublepage\phantomsection\pdfbookmark{\abstractname}{abstract} %% fuegt ersten Abstract in die Bookmarks ein

\begin{otherlanguage}{ngerman}   %... falls Restdokument auf English...
\begin{abstract}
  Endoskopische Untersuchungstechniken spielen in der modernen Medizin eine zunehmend wichtigere Rolle. Das Ziel der Endoskopie ist dabei möglichst umfangreich Informationen für eine verlässliche Diagnose zu sammeln. Durch die Kombination verschiedener optischer Abbildungstechniken in einem Endoskop wird aktuell versucht die Möglichkeiten der endoskopbasierten Diagnostik zu erweitern. Die Kombination einer mikroskopischen Abbildung und der optischen Kohärenztomographie (OCT) in einem einzigen Endoskop zur Verbesserung der Diagnostik scheint hierbei ein vielversprechender Ansatz, da hierdurch zusätzliche Informationen über die Beschaffenheit des untersuchten Gewebes erlangt werden können.

Das Ziel dieser Arbeit ist die Realisierung eines neuartigen Faser-Scanners für dreidimensionale OCT, dessen Design dahingehend optimiert wird, dass er zusammen mit einer Mikroskopiemodalität in einem Messkopf integriert werden kann. Durch die Verwendung eines piezolektrischen Röhrchens wird ein Faser-Scanner implementiert, der das zweidimensionale Scanmuster erzeugen kann, dass zur Realisierung von dreidimensionalen OCT-Aufnahmen benötigt wird. Der in dieser Arbeit entwickelte Scanner hat einen Außendurchmesser von 0,9 mm und eine Länge von 9 mm. Zur Kontaktierung der vier Außenelektroden des piezoelektrischen Röhrchens werden selbstentwickelte, 10 µm dicke Polyimid Flachbandkabel verwendet. Die in dieser Arbeit präsentierte OCT-fähige Demonstratorsonde ist, soweit dies durch den Autor verifiziert werden konnte, eine der kompaktesten Implementierungen eines OCT-fähigen Mikroendoskops. 

Die Ergebnisse, die mit der in dieser Arbeit entwickelten OCT-fähigen Demonstratorsonde erhalten wurden, zeigen eine hohe Übereinstimmung mit den aufgestellten analytischen Modellen und Simulationen. Aus diesem Grund könnte der in dieser Arbeit beschriebene Ansatz als Vorlage für verschiedenste Sonden mit scannerbasierten Abbildungstechniken herangezogen werden.
	\bigskip\par
  \textbf{Stichwörter:} Endoskopische Sonde, bimodal, Optische Kohärenztomographie, Faser-Scanner
\end{abstract}
\end{otherlanguage}

%%%%%%%%%%%%%%%%%%%%%%%%%%%%%%%%%%%%%%%%%%%%%%%%%%%%%%%%%%%%%%%%%%%%%%%%%%%%%%%%%
%%%%%                                                                 English Abstract												
%%%%%%%%%%%%%%%%%%%%%%%%%%%%%%%%%%%%%%%%%%%%%%%%%%%%%%%%%%%%%%%%%%%%%%%%%%%%%%%%%
%%%%%
%%%%%
%\begin{otherlanguage}{english}
\begin{abstract}
  Endoscopic methods play an increasingly important role in modern medicine. In general, the goal of endoscopy is to obtain as much information as possible for a viable diagnosis. To further improve diagnostics, there is a clear tendency towards multi-modal imaging by combining different imaging techniques. Making use of microscopy and Optical Coherence Tomography (OCT) in a single device is a promising approach to gather additional information and to thereby improve clinical diagnostics. 

The main goal of this work is the realization of a novel fiber scanner for 3D OCT imaging, designed for its combination with a full field microscope to form a multimodal probe. A tubular piezoelectric fiber scanner is used to perform en face scanning required for 3D OCT measurements. The complete scanning engine has an outer diameter of \SI{0.9}{\milli\meter} and a length of \SI{9}{\milli\meter}, and features custom fabricated \SI{10}{\micro\meter} thick polyimide flexible interconnect lines to address the four piezoelectric electrodes. To the best of our knowledge, the presented probe is one of the most compact implementation of an OCT microendoscope.

Optical and mechanical characterization of the scanner, implemented in a single modality demonstrator probe, show a high correlation with the analytical calculations and simulations. Thus, the approach described in this work constitutes a blueprint for a wide range of future scanning imaging probes.

\bigskip\par
  \textbf{Keywords:} Endoscopic probe, bimodal, optical coherence tomography, fiber scanner.
\end{abstract}
%\end{otherlanguage}