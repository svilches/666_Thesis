%%%%%%%%%%%%%%%%%%%%%%%%%%%%%%%%%%%%%%%%%%%%%%%%%%%%%%%%%%
%%                                                                                                             		                                      %%
%%                                                Personal settings																				%%
%%                                                                                                                                                       %%
%%%%%%%%%%%%%%%%%%%%%%%%%%%%%%%%%%%%%%%%%%%%%%%%%%%%%%%%%%

\usepackage{graphicx}   % Graphikpaket (fuer PNG,GIF,JPG)

\usepackage{subfig}       % Notwendig f�r die Verwendung von \subfigure

\usepackage{float,rotating}% u.a. Option [H] fuer fixe Position; Drehung von Abbildungen etc.

\usepackage{pdfpages,afterpage}% zum Einbinden ganzer pdf-Seiten - Befehl: \includepdf[pages={1,3-5}]{Dateiname}

\usepackage{array}      % Befehle fuer erweitertes Tabellenlayout
\usepackage{booktabs}   % unterschiedliche horizontale Linien in Tabellen: \toprule \midrule \bottomrule

\usepackage{icomma}  % Behebt folgendes Problem: "Immer wenn man im Mathematikmodus eine Dezimalzahl darstellen will also z.B. 5,5 macht er daraus 5, 5 (mit Leerzeichen zwischen, und 1. Nachkommastelle), was ziemlich unsch�n ist."

\usepackage[space]{grffile} % The pack­age ex­tends the file name pro­cess­ing of pack­age graph­ics to sup­port a larger range of file names. For ex­am­ple, the file name may con­tain sev­eral dots. Or in case of pdfTEX in PDF mode the file name may con­tain spaces.

\usepackage{siunitx}

\usepackage{bm}   % Bold math

\makeatletter
\setlength{\@fptop}{0pt}
\setlength{\@fpbot}{0pt plus 1fil}
\makeatother
